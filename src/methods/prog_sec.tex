\section{Programming Security}

\subsection{Challenge-Response}\label{chalres}
You insert a car key into a car engine. The engine would now like to know
that this key is valid.

Let $K$ be the key, and $E$ the engine. Then,:
\begin{gather}
	\begin{align}
		&E \rightarrow K: N \\
		&K \rightarrow E: \{E, N\}_{K}
	\end{align}
\end{gather}

Here, N is a challenge sent by the engine to the key in step (1).
The key responds by encrypting the text. Iff the engine can decrypt
and get valid number, will it start.


\subsection{Needham-Schroeder Protocol}
A primitive, three-way~\nameref{keydistribution} process. Also known
to inspire Kerberos%
\footnote{originally from Greek/Roman mythology, a three-headed dog, or "hellhound",
	which guards the entrance of Hades.},
the security protocol in windows. The word \textit{primitive} is used to 
emphasize that the protocol comes from an age where attacks were
mostly thought of as external-to-internal penetrations, i.e.\
one needs to assume that attacks come from external networks,
by people without means of authentication. Hence, the protocol
fails to defend against users of it's own system.

Having defined key-distribution in section~\ref{secword}, the following should 
be somewhat intuitive:
\setcounter{equation}{0}
\begin{gather}
	\begin{align}
		&A \rightarrow S: A,B,N_{A}  \\
		&S \rightarrow A: \{N_A, K_{AB}, B, \{K_{AB}, A\}_{K_{BS}}\}_{K_{AS}} \\ 
		&A \rightarrow B: \{K_{AB}, A\}_{K_{BS}} \\ 
		&B \rightarrow A: \{N_B\}_{K_{AB}} \\
		&A \rightarrow B: \{N_B-1\}_{K_{AB}}
	\end{align}
\end{gather}
The difference from plain~\nameref{keydistribution} is that we here introduce the
concept of \nameref{nonce}s.
In step 1, Alices attaches her nonce $N_{A}$ when talking to Sam.
The nonce will be used in Sam's reply. That way, Alice can know that
she is not recieving an old/corrupted message.

The entitiy $\kappa = $ $\{K_{AB}, A\}_{K_{BS}}$ represents the the shared key 
between Alice and Bob, encrypted such that only Bob can read it (denoted by the
subscript $_{K_{BS}}$).
$K_{AB}$ is the key, and the $A$ is included such that Bob knows it is 
inteded for his use to Alice.
In step 3, Alice sends $\kappa$ to Bob. Bob can decrypt with his key, given
to him earlier by Sam. Finally, to ensure that no data has been corrupted
by Eve or other errors, the last two steps verify the legimiticay of the keys.
Should Alice fail to return back $N_{B} - 1$, that is, a nonce from Bob - 1,
then Bob can know that he is not talking to a valid user.

One easy to understand flaw in this system is that if someone is able 
to obtain Alice's identifier, they could easily impersonate Alice. 
The only required step is to contact Sam, get keys, and then communicate.
Alice cannot be aware of this since she never initiated the requests.
To revoke the damage done, Sam would have to keep logs of every issued key
and thereafter alert everyone who has recieved Alice's key that she 
was compromised.

