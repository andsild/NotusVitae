\section{Calculus}

\begin{definition}[Affine]
    $ 
    f(x_{1}, x_{2}, \dots, x_{n}) = 
    a_{1}x_{1}, a_{2}x_{2}, \dots a_{n}x_{n}
    $
\end{definition}

\begin{definition}[arcsin]
    \begin{align*}
        \sin{y} &= x \\
        \arcsin{x} &= \sin^{-1}{x} = y
    \end{align*}

    Properties:
    \begin{itemize}
        \item $\arcsin{x} = \frac{\pi}{2} - \arccos{x} = 90º - \arccos{x}$
        \item $\cos{(\arcsin{x}} = \sin{(\arccos{x})} = \sqrt{1-x^{2}}{}$
        \item $ x = -1 \rightarrow \arcsin{x} = -1\times\frac{\pi}{2}$
        \item $ x = 1 \rightarrow \arcsin{x} = \frac{\pi}{2}$
        \item $ x = 0 \rightarrow \arcsin{x} = 0$
    \end{itemize}
\end{definition}

\begin{definition}[arccos]\label{arccos}
    Properties:
    \begin{itemize}
        \item $ x = -1 \rightarrow \arccos{x} = \pi$
        \item $ x = 1 \rightarrow \arccos{x} = 0$
        \item $ x = 0 \rightarrow \arccos{x} = \frac{\pi}{2}$
    \end{itemize}
\end{definition}

\begin{definition}[Arc length]
    Length of a curve when straightened out.
\end{definition}
\begin{definition}[arccos]
    While cosine shows the relation between lengths in a triangle, 
    arccos gives the angle.
\end{definition}

\begin{definition}[Arithmetic-geometric mean inequality]\label{arigeo}
    $
    \newline {(\prod\limits_{i = 1}^{k} a_{i})}^{1/k}
    \leq {1 \over k} \sum\limits_{i = 1}^{k} a_{i}
    $
\end{definition}

\begin{definition}[Bijection]
    \begin{align}
        S,R \text{\ are sets} \\
        \forall{i \in S}, \exists!{f(i) \in R} \wedge \\
        \forall{i \in R}, \exists!{f(i) \in S} \\
    \end{align}
\end{definition}

\begin{definition}[Boundary value problem]
    A differential equation with additional restrains. A solution to a boundary
    value problem is a solution to the differential equation which also
    satisfies the boundary conditions.
    
\end{definition}

\begin{definition}[Convolution]
    Dictionary: a coil or twist, especially one of many.\newline
    Informal: an expression of how a shape from one function is modified by 
        the other.

    Can also be used to smoothen a discontinous a function (making it continous 
    on the given range). We need to normalize our g(x - $\tau$) so that we do
    not continously increase the f(x)
\end{definition}

\begin{definition}[Continuous]
In mathematics, a continuous function is a function for which
"small" changes in the input result in "small" changes in the output.

As an example, consider the function h(t), which describes the height of a
growing flower at time t. This function is continuous. By contrast, if M(t)
denotes the amount of money in a bank account at time t, then the function
jumps whenever money is deposited or withdrawn, so the function M(t) is
discontinuous.

\end{definition}

\begin{definition}[Continuous variables]
    There are three types of continuous variables:
    \begin{center}
    \begin{description}
        \item[Nominal] Categorized within groups: box 1, 2, 3 or 4.
        \item[Dichotomous] Boolean, yes or no.
        \item[Ordinal] A nominal variable, just that different groups give
            different values. E.g.\ to like something on a scale of 1 to 10
            gives you a ordinal range
    \end{description}
\end{center}
\end{definition}

\begin{definition}[Concave]
    Let $f$ be a function defined on the interval $[x_{1}, x_{2}]$.
    This function is concave according to the definition if, for every pair of
    numbers a and b with $x_{1} \leq a \leq x_{2}$ and $x_{1} \leq b \leq
    x_{2}$, the line segment from $(a,  f (a))$ to $(b,  f (b))$
    lies on or below the function.

    \begin{itemize}
        \item The sine function is concave on the interval $[0, \pi]$
        \item Concave if every line segment joining two point is never above
              the graph 
        \item Concave functions has $f\prime\prime(x) \leq 0$ in a given
              interval 
    \end{itemize}
\end{definition}

\begin{definition}[Congruence]
    Similar shape and growth, just different scalar / rotation.
\end{definition}

\begin{definition}[Differential operator]
    An operator to do differentiation. This is mainly to abstract
    differenentiation.

\end{definition}

\begin{definition}[Discretization]
    In mathematics, discretization concerns the process of transferring
    continuous models and equations into discrete counterparts. 
    AKA\ smoothening curves to avoid jumps.

\end{definition}

\begin{definition}[Divergence]\label{divergence}
    measures the magnitude of a~\nameref{vectorfield}'s source or sink at a
    given point, in terms of a signed scalar.

    E.g.\ air can be thought of to have a point $s \text{ and } t$, where they
    push out hot and cool air, respectively. From these points, you can create
    a~\nameref{vectorfield} that shows how air spreads from $s \text{ and }
    t$. The divergence measures the collected value from each of 
    these~\nameref{vectorfield}s.

    The operator for divergence is \verb|{div}|.

    Given vector field $F = Ui, Vj, Wk$:
    \begin{align}
            div \textbf{F} = \nabla \cdot F = 
            \frac{\partial{U}}{\partial{x}}+
            \frac{\partial{V}}{\partial{y}} +
            \frac{\partial{W}}{\partial{z}}
    \end{align}

\end{definition}

\begin{definition}[Elementary function]
    In mathematics, an elementary function is a function of one variable built
    from a finite number of exponentials, logarithms, constants, and $nth$ roots
    through composition and combinations using the four elementary operations
    $(+ – × ÷)$.

\end{definition}

\begin{definition}[Extreme point]
    A point furthest away from something.
\end{definition}

\begin{definition}[Field]
    A physcial quantity that has a value for each point in space and time.
\end{definition}

\begin{definition}[Finite difference]
    The difference in a function for $f(x + a) - f(x + b)$.

    There are three types:
    \begin{description}
        \item[Forward] is of the form $\bigtriangleup{f}(x) = f(x + h) - f(x)$
        \item[Backward] is of the form $\nabla{f}(x) = f(x) - f(x - h)$
        \item[Centered] is of the form $\delta{f}(x) = f(x + \frac{1}{2}h) - f(x - \frac{1}{2}h)$
    \end{description}
\end{definition}

\begin{definition}[Flow]
    Motion of particles in a given set.
\end{definition}

\begin{definition}[Generalized function]
   A distribution without steps, i.e.\ it is continious. 
\end{definition}

\begin{definition}[Harmonic numbers]
    $H_{n} = 1 + \frac{1}{2} + \frac{1}{3} + \dots + {1}{n} =
    {\sum\limits_{k = 1}^{k}} \frac{1}{k} \simeq \ln n$
\end{definition}

\begin{definition}[Heavyside Step Function]
    $$
    H(x) = \left\{
            \begin{array}{l l}
                0 & \text{for} x < 0 \\
                \frac{1}{2} & x = 0 \\
                1 & \text{for} x > 0 \\
            \end{array}
        \right.
    $$
\end{definition}

\begin{definition}[Hyperbolic]
    Two curves that kind of face each other like bananas.
\end{definition}

\begin{definition}[Identity function]
    $\forall{x}, f(x) = x$
\end{definition}

\begin{definition}[Integral]
    The ``reverse'' to a derivate. 

    \begin{align}
        \int_{a}^{b}x^{n} dx = \frac{x^{n+1}}{n+1} + C \\
    \end{align}

\end{definition}

\begin{definition}[Law of cosines]
    \begin{align*}
        v \cdot u = |v||u|\cos{\theta} \\
    \end{align*}
    Note that if $v, u$ are unit vectors, their lengths are both 1.
    We can then rewrite the expression as
    \begin{align*}
        v \cdot u &= \cos{\theta} \\
        \arccos{(v \cdot u)} &= \theta
    \end{align*}
\end{definition}

\begin{definition}[Laplace operator]
    Transform a function of $f$ of $t$ to a function $f$ of $s$

    A differential operator given by the~\nameref{divergence} of a gradient 
    function in euclidian space.
\end{definition}

\begin{definition}[Laplacian Matrix]
    sometimes called admittance matrix, Kirchhoff matrix or discrete Laplacian,
    is a matrix representation of a graph.

\end{definition}

\begin{definition}[Lattice]
    In mathematics, especially in geometry and group theory, a lattice in
    $\mathbf{R}^n$ is a discrete subgroup of $\mathbf{R}^n$ which spans the real
    vector space $\mathbf{R}^n$. Every lattice in $\mathbf{R}^n$ can be generated
    from a basis for the vector space by forming all linear combinations with
    integer coefficients. 

\end{definition}

\begin{definition}[Line integral]
    An integral where the function to be integrated is along a curve.
    The function is usually a~\nameref{vectorfield} or~\nameref{scalarfield}.

\end{definition}

\begin{definition}[Parabola]
    A U-shaped, 2d, symmetrical curve.
\end{definition}

\begin{definition}[Parameterization]
    Represent a curve as a function.
    \begin{align}
        x = \cos{t} \\
        y = \sin{t}
    \end{align}
    is the parametric representation of a unit circle.

\begin{definition}[Partial derivative]
    Given a function $f$ with multiple parameters, a partial derivative is a
    derivative with respect to one of those variables.

    Partial derivatives are often denoted by $\partial$.

    To use partial derivation, you often assume that the other variables are 
    constants. Otherwise, there is an infinite number of tangent lines at
    any point, so you will have to have some range on the tangents.

\end{definition}

\end{definition}

\begin{definition}[Partial differential equation]
    A set of variables, and equations that show how they are all linked 
    together.
\end{definition}

\begin{definition}[Rectangle function]
    $$
    \Pi(x) = \left\{
            \begin{array}{l l}
                0 & \text{for} x > \frac{1}{2} \\
                \frac{1}{2} & \text{for} x = \frac{1}{2} \\
                1 & \text{for} x < \frac{1}{2} \\
            \end{array}
        \right.
    $$
\end{definition}

\begin{definition}[Riemanns sum]
    Sum from an integral. Divide the area under/over a curve into rectangles
    or trapezoids, and sum together their area. The smaller the shapes, the
    better.
\end{definition}

\begin{definition}[Standard form]
    To write a number as a power of 10.
\end{definition}

\begin{definition}[Vector Field]\label{vectorfield}
    An assigment of direction for a given set of points in an euclidian space.
    E.g.\ select every (10n, 10n) pixels in an image and get their derivative.
\end{definition}
