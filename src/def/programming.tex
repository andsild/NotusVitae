\section{C++}

const \textless{C}\textgreater\& function \{\ return FOO; \}
means it will return a constant reference to C

\textless{C}\textgreater\& function const \{\ return FOO; \}
means it will not modify any data in C.

const char *var is a pointer to a constant value, not a const pointer
(char const *var).


std::vector\textless{T}\textgreater will typically allocate space in $2^{n}$, where n is minimal.
Whenever n increases, a new buffer is allocated, which means that it will
have to iterate over all it's former elements - giving ${2^n+1}$ iterations.
Hence, one should pre-allocate the space needed when possible.

Whenever you use cout to output something, ostream buffers it but does not send
it to the output device immediately. Only when the buffer is flushed will the
output get sent to the destination. This can cause trouble if e.g.\
twidth he buffer is full (usually 512 bytes).

\begin{definition}[Rule of three]
    If you implement at least one of the following, you should implement
    the others, too:
    \begin{itemize}
        \item Destructor 
        \item Copy constructor 
        \item Copy assignment operator 
    \end{itemize}
\end{definition}

\section{Haskell}
\begin{definition}[=\textgreater]
    Every expression preceeding the =\textgreater is a 
    \textbf{class constraint:} it forces types and memberships

\end{definition}

\begin{definition}[-\textgreater]
    Used to separate variables and return types in functions.
\end{definition}

\begin{definition}[Guards]
    Guards are indicated by pipes that follow a function's name and its
    parameters. They evaluate to true or false, and then the following
    function body is used in case of true.
\end{definition}

\begin{definition}[return]
    Turn a type into IO.\
    E.g.
    \begin{minted}{haskell}
    getFilename :: String -> IO String
    getFilename file = do
        bool <- doesFileExist file
        if bool
        then return file
        else return "batman.wav" 
    \end{minted}

    Where the safe ``file'' is returnned as IO String
\end{definition}


\section{More words and def\dots}


\begin{definition}[Concurrency]\label{concurrency}
    To execute several unrelated tasks at the same time. E.g.\ on a game-server
    one thread deals with chatting, one deals with connections, etc.
    One important aspect of concurrent threads is that they
    are~\nameref{nondeterministicprog}. That means that you can predict their
    state pre-emptively. That is also why we usually do so much thread joining
    etc.
    See also~\nameref{parallelism}.

\end{definition}

\begin{definition}[Currying]
    the technique of translating the evaluation of a function that takes
    multiple arguments (or a tuple of arguments) into evaluating a sequence of
    functions, each with a single argument (partial application)

    E.g.\ $f(x,y) ---> h(x) = y -> f(x, y)$.
    Here, $h(x)$ is a curried version of f, and the $->$ is a function that
    maps the result from $h$ to $f$.
\end{definition}

\begin{definition}[Folding]
    Take a list, apply a function and reduce the list to a single value.
    (this is the python's reduce)

\end{definition}

\begin{definition}[High order functions]
    can take functions as parameters and return functions as return values.

\end{definition}

\begin{definition}[Include guards]
    Avoid re-invoking a header multiple times: enforce that they are only 
    loaded once. In C++ (11) you can use the 
    \begin{verbatim}#pragma once\end{verbatim}
\end{definition}

\begin{definition}[Nondeterministic programming]\label{nondeterministicprog}
    A nondeterministic programming language is a language which can specify, at
    certain points in the program (called "choice points"), various alternatives
    for program flow. Unlike an if-then statement, the method of choice between
    these alternatives is not directly specified by the programmer; the program
    must decide at run time between the alternatives, via some general method
    applied to all choice points

\end{definition}

\begin{definition}[Lock order inversion]
    Entity A ackquires a lock on $X$, entity B ackquires a lock on $Y$.
    Now, $X$ depends on something in $Y$ and verca visa. The programs will
    now deadlock - each waiting for the other to finish their task.

\end{definition}

\begin{definition}[Parallelism]\label{parallelism}
    To use multiple threads to solve one problem. See
    also~\nameref{concurrency}.

\end{definition}


\begin{definition}[Strict variable]
    A variable with a determined typed. E.g.\ writing "int myvar = 5" instead
    of just "myvar = 5" (non-strict).

\end{definition}
