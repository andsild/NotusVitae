\section{Algorithms}\label{section:word}

\begin{definition}[Approximation algorithm]
    Find approximate solutions to~\nameref{optproblem}
\end{definition}

\begin{definition}[Asymptotic Polynomial-time Approximation Scheme, APTAS]
\label{APTAS}
    A family of algorithms, and a constant $c$ such that all solutions has an
    approximation of $(1 + \epsilon)\times{OPT} + c$ for minimization problems.
\end{definition}


\begin{definition}[Bin-packing]
    A series of algorithms to learn how to distribute $n$ numbers into $k$ bins.
    First-fit, best-fit, worst-fit (stack into where there is most free space),
    best-fit, etc.
\end{definition}

\begin{definition}[Complimentary slackness]
    Given an optimal solution to a linear program,
    $Z_{LP}$ and it's dual $Y_{LP}$, with $x_{1}, x_{2}, \dots x_{n}$ and
    $y_{1}, y_{2}, \dots y_{n}$ respecively, with
    $w_{1}, w_{2}, \dots w_{n}$ and $z_{1}, z_{2}, \dots z_{n}$  as slack
    variables for each solution, respectively,
    then $\forall x, x_{i}z_{i} = 0$ and $\forall y, y_{i} w_{i} = 0$

    This necessary condition for optimality conveys a fairly simple economic
    principle.  In standard form (when maximizing), if there is slack in a
    constrained primal resource (i.e., there are "leftovers"),
    then additional quantities of that resource must have no value. 
\end{definition}

\begin{definition}[Difference heuristic and approximation]
    While an heurisitc makes a choice, without a guarantee of optimality,
    an approximation can make a choice and know that this choice will render
    a solution within a factor of OPT.
\end{definition}

\begin{definition}[EDD]\label{EDD}
    Earliest Due Date
\end{definition}

\begin{definition}[F-approximation]
    Also referred to as a linear approximation, using a function f,
    which is affine.
\end{definition}

\begin{definition}[FPTAS, fully polynomial approximation scheme]\label{FPTAS}
    As in~\nameref{PTAS}, just $\frac{1}{\epsilon}$.
    The algoritm is required to be polynomial both in running time and problem
    size.

    Note that strongly NP-complete problems do not have any FPTAS.
\end{definition}

\begin{definition}[Integrality gap]\label{integralitygap}
    The biggest difference between an IP and LP
\end{definition}

\begin{definition}[Makespan]
    The total length of a schedule; from 0 to $C_{max}$
\end{definition}

\begin{definition}[$\tilde{O}$]\label{otilde}
    Given function $f(x)$, $\tilde{O}(f(x)) = O(f(x)\cdot{\log^{k}{f(x)}}$
\end{definition}

\begin{definition}[Optimization problem]\label{optproblem}
    To find the best solution of $n$ feasible solutions.
\end{definition}

\begin{definition}[Perfect matching]
    A collection $E^{\prime} \subseteq E$ of edges in a graph 
    $G = (V,E)$, such that $\forall v \in V$, are connected 
    from $E^{\prime}$ only once.
\end{definition}


\begin{definition}[Pre-empty schedule]\label{pre-emptive}
    You can interrupt task and re-continue them.
\end{definition}


\begin{definition}[PTAS, Polynomial-time approximation scheme]\label{PTAS}
    Given an optimization problem (e.g.\ an NP-problem) and a parameter 
    $\epsilon$, produce a solution within (($1 + \epsilon) \times OPT$)
    $\epsilon > 0$

    E.g.\ for the traveling salesman, a tour would be of length max 
    $(1 + \epsilon) \times L$, with $L$ being the length of the tour

    Note that for minimization, there is $1 + \epsilon$, and for maximization,
    there is $1 - \epsilon$
    
    If you have a scheme with $(1 \pm \epsilon) \times OPT + \kappa$, then 
    it is not under PTAS.\ PTAS only handles the former part, $(1 \pm \epsilon)$

    For MAX SNP, there does not exist polynomial approximation schemes
\end{definition}

\begin{definition}[$\rho$-approximation]
    Polynomial algorithm that is guaranteed to have objective function
    to OPT within $\rho$ of
    optimum (not the ($1 + \epsilon$) of~\nameref{PTAS}).
\end{definition}

\begin{definition}[Scheduling]
    See~\nameref{srpt},
    \begin{itemize}
        \item \textbf{$P_{i}$} = time to do a job $i$
        \item \textbf{$R_{i}$} = earliest time a job $i$ can start
        \item \textbf{$C_{i}$} = time of completion for job $i$
        \item \textbf{$D_{i}$} = due date for job $i$
        \item \textbf{$L_{i}$} = $C_{i} - D_{i}$
    \end{itemize}
\end{definition}

\begin{definition}[Strong duality]
    The optimal value of the dual is equal to that of the primal linear program.

    $ \sum{y^{*}_{i}} = \sum{w_{i}x_{i}}$
\end{definition}

\begin{definition}[Weak duality property]
    No dual program has a solution greater than the optimal of the primal
    linear program
\end{definition}

\begin{definition}[$\alpha$ approximation]
    Produce a solution who's value is within a factor of $\alpha$ 
    of the optimal.
\end{definition}
