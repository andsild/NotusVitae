\section{Image Processing}

\begin{definition}[Anisotropy]\label{anisotropy}
    Directionally dependent. Opposite of~\nameref{isotropy}.
    Example is to apply a filter to a set of images - consider
    what would happen if one image was tilted.
\end{definition}

\begin{definition}[Anisotropic diffusion]
    also called Perona-Malik diffusion, is a technique aiming at reducing image
    noise without removing significant parts of the image content, typically
    edges, lines or other details that are important for the interpretation of
    the image.

\end{definition}

\begin{definition}[extrapolation]
    The process of estimating, beyond the original observation range, the value
    of a variable on the basis of its relationship with another variable.

    Creating a tangent line at the end of the known data and extending it
    beyond that limit.

\end{definition}

\begin{definition}[Gaussian blur]
    Use a gaussian function on an image to reduce image noise and detail
\end{definition}

\begin{definition}[Gaussian filter]
    Gaussian filters have the properties of having no overshoot to a step
    function input while minimizing the rise and fall time.

\end{definition}


\begin{definition}[Gradient flow]
    $$
        V = \nabla{f} = \left(\
        \frac{\partial{f}}{\partial{x_{1}}},
        \frac{\partial{f}}{\partial{x_{2}}},
        \dots,
        \frac{\partial{f}}{\partial{x_{n}}}
    \right)
    $$
\end{definition}

\begin{definition}[Grayscale]
    an image in which the value of each pixel is a single sample, that is, it
    carries only intensity information

    Grayscale images are distinct from one-bit bi-tonal black-and-white images,
    which in the context of computer imaging are images with only the two
    colors, black, and white (also called bilevel or binary images). Grayscale
    images have many shades of gray in between.

    Grayscale images are often the result of measuring the intensity of light
    at each pixel in a single band of the electromagnetic spectrum 

\end{definition}

\begin{definition}[Image gradient]
    Gradual blend of color
\end{definition}

\begin{definition}[Image noise]
    Unwanted signal, electrical signals not wanted in an image
\end{definition}

\begin{definition}[Image masking]
    Apply an image over another. This can be used for e.g.\ putting a square image
    with only contents in the middle over another.
\end{definition}

\begin{definition}[Isotropy]\label{isotropy}
    Directionally independent. Opposite of~\nameref{anisotropy}.
\end{definition}

\begin{definition}[Laplacian]
    Highlight regions of rapid intensity change and is therefore often used for
    edge detection.

    Often applied to an image that has first been smoothed with something
    approxiating in order to reduce its sensitivity to noise.

\end{definition}


\begin{definition}[SRGB]
    sRGB is a standard RGB color space created cooperatively by HP and
    Microsoft in 1996 for use on monitors, printers and the Internet.

\end{definition}


\begin{definition}[Poisson image editing]
    Blend two images together and make them seem alike.
\end{definition}

\begin{definition}[Raster order]
    Begin at top left, proceed to right, then at leftmost pixel in next line.
\end{definition}

\begin{definition}[Raster image]
    An image with pixels.
\end{definition}

\begin{definition}[Scalar field]\label{scalarfield}
    Associate a value to every point in a space. E.g.\ for an image, you
    could assign each pixel a color value.
\end{definition}

\begin{definition}[Stencil]
    A figure that connects dots on a grid. Usually a cross-like figure 
    (one center dots, and one edge out in 4 directions from the center to
    other dots).
\end{definition}
