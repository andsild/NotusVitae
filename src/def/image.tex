\section{Image Processing}

\begin{definition}[Aliasing]
    Images are constructed and then reconstructed one or more times - 
    each of these reconstructions are referred to as aliases.
\end{definition}

\begin{definition}[Anisotropy]\label{anisotropy}
    Directionally dependent. Opposite of~\nameref{isotropy}.
    Example is to apply a filter to a set of images - consider
    what would happen if one image was tilted.
\end{definition}

\begin{definition}[Anisotropic diffusion]
    also called Perona-Malik diffusion, is a technique aiming at reducing image
    noise without removing significant parts of the image content, typically
    edges, lines or other details that are important for the interpretation of
    the image.

\end{definition}

\begin{definition}[Aperture]
    A hole which light goes through.
\end{definition}

\begin{definition}[Convolution]\label{convolution}
    In image-processing, this can in general be thought of as applying a mask
    that takes the local neighbourhood for a pixel x and calculates its
    gradient.

    Noteworthingly, convolution as an operation to an image is in general
    $O(n^{2})$.
\end{definition}

\begin{definition}[Curve representation]
    How to represent a curve: computational cost is crucial.

    \begin{description}
        \item[Explicit] $y = f(x)$ - only lines but really easy to genereate.
        \item[Implicit] $f(x,y) = 0$. \dots imagine a circle: to know if a point
            is on the line, we can use pythagoras's $x^{2} + y^{2} - r^{2} = 0$. Costs a lot.
        \item[Parametric] $(x,y) = ( f(u), g(u))$. Oh yeah..
    \end{description}
    
\end{definition}

\begin{definition}[Dilation (of functions)]
    To skew a function, e.g.\ for $f(x) = x$, you skew to $\hat{f}(x) = x + 2$.
\end{definition}

\begin{definition}[Dirichet boundary condtions]
    In image processing, you can say that this boundary condition is to assume
    for a set of differential equations, the unknowns at the borders are 
    0: $u(0) = u(1) = 0$.

\end{definition}

\begin{definition}[extrapolation]
    The process of estimating, beyond the original observation range, the value
    of a variable on the basis of its relationship with another variable.

    Creating a tangent line at the end of the known data and extending it
    beyond that limit.

\end{definition}

\begin{definition}[Filter]
    Applying a mask or operation to an image to extract or distort values.

    \begin{description}
        \item[High-pass] takes only the high frequency into play, lower values
            are not affected that strongly (if at all)
        \item[Low-pass] \dots
    \end{description}
\end{definition}

\begin{definition}[Gaussian blur]
    Use a gaussian function on an image to reduce image noise and detail
\end{definition}

\begin{definition}[Gaussian filter]
    Gaussian filters have the properties of having no overshoot to a step
    function input while minimizing the rise and fall time.

\end{definition}

\begin{definition}[Gaussian pyramid]
    A stack of images, where for each iteration, you take neighboring pixels
    and make them into one. That is, you reduce the intensity in the picture.
    This compresses the image. 

    It is a pyramid because for each iteration you take neighboring pixels
    from the previous pyramid to generate the current.
\end{definition}


\begin{definition}[Gradient flow]
    $$
        V = \nabla{f} = \left(\
        \frac{\partial{f}}{\partial{x_{1}}},
        \frac{\partial{f}}{\partial{x_{2}}},
        \dots,
        \frac{\partial{f}}{\partial{x_{n}}}
    \right)
    $$

    Note that for an oridnary image, the gradient at a given position is 
    given by:
    $$
    \nabla{f} = 
    \frac{\partial{f}}{\partial{x}}\hat{x} +
    \frac{\partial{f}}{\partial{y}}\hat{y}
    $$
    Where the first (x) term is the gradient in direction x, and y is the
    gradient in direction y.

    One can also calculate the direction of the gradient using
    $\theta = a\tan^{2}{\frac{\partial{f}}{\partial{y}}, \frac{\partial{f}}{\partial{x}}}$
\end{definition}

\begin{definition}[Grating]
    A collection of identical pararell objects, placed next to each other with
    equal spacing.
    \end{definition}
\begin{definition}[Grayscale]
    an image in which the value of each pixel is a single sample, that is, it
    carries only intensity information

    Grayscale images are distinct from one-bit bi-tonal black-and-white images,
    which in the context of computer imaging are images with only the two
    colors, black, and white (also called bilevel or binary images). Grayscale
    images have many shades of gray in between.

    Grayscale images are often the result of measuring the intensity of light
    at each pixel in a single band of the electromagnetic spectrum 

\end{definition}

\begin{definition}[Image gradient]
    Gradual blend of color
\end{definition}

\begin{definition}[Image noise]
    Unwanted signal, electrical signals not wanted in an image
\end{definition}

\begin{definition}[Image masking]
    Apply an image over another. This can be used for e.g.\ putting a square image
    with only contents in the middle over another.
\end{definition}

\begin{definition}[Interpolation]
    A way to estimate (pixel) values when resizing to a larger image.

    An example from calculus is: given temps at noon and midnight, estimate
    the temp at 5 PM as the mean of noon and midnight.

    Interpolation can also (in signal processing) be referred to as 
    \textbf{upsampling}.

    Interpolation can be \textit{linear}(1D), \textit{bilinear}(2D) and so on\dots

\end{definition}

\begin{definition}[Isotropy]\label{isotropy}
    Directionally independent. Opposite of~\nameref{anisotropy}.
\end{definition}

\begin{definition}[Laplacian]
    Highlight regions of rapid intensity change and is therefore often used for
    edge detection.

    Often applied to an image that has first been smoothed with something
    approxiating in order to reduce its sensitivity to noise.

\end{definition}

\begin{definition}[Spatial frequency]
    The number of changes in color values that occur per space.

    Images with high spatial frequency are detailed, images with low spatial 
    frequency will appear blurred.

    I.e.\ sharp transitions from low/high to high/low intensities are said
    to have a large spatial frequency,

\end{definition}


\begin{definition}[SRGB]
    sRGB is a standard RGB color space created cooperatively by HP and
    Microsoft in 1996 for use on monitors, printers and the Internet.

\end{definition}


\begin{definition}[Poisson image editing]
    Blend two images together and make them seem alike.
\end{definition}

\begin{definition}[Raster order]
    Begin at top left, proceed to right, then at leftmost pixel in next line.
\end{definition}

\begin{definition}[Raster image]
    An image with pixels.
\end{definition}

\begin{definition}[Ringing]
    The size of the oscillations after a peak frequency. E.g.\ if you have a 
    sudden overshoot (bright color) the amplitude of the next wave is
    the ringing. (???)
\end{definition}

\begin{definition}[Scalar field]\label{scalarfield}
    Associate a value to every point in a space. E.g.\ for an image, you
    could assign each pixel a color value.
\end{definition}

\begin{definition}[Seperable filter]
    A filter that can be written as the product of two or more small filters.
    This is usually done to reduce the computational costs.
    E.g.~\nameref{convolution} is usually faster in 1d than iterating $nD$, for
    $n \ge 2$.
\end{definition}

\begin{definition}[Spline]
    A line that goes through a series of points.

    The section between the lines is interpolated. This can be done linearly,
    quadratic, \dots so on.
\end{definition}

\begin{definition}[Stencil]
    A figure that connects dots on a grid. Usually a cross-like figure 
    (one center dots, and one edge out in 4 directions from the center to
    other dots).
\end{definition}
