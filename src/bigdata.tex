\section{Big Data}
\begin{definition}[CAP theorem]\label{def:captheorem}
    Working with Big Data involves dealing with inputs so large that conventional methods do not work.
    This also involves that we have to compromise, not opting for ``optimal'' solutions like RDMS.\
    We have 
    \begin{description}
        \item[Consistency:] Whether or not copies of data are the same across all nodes.
        For example, having a server in London with data X and another server in USA with data X', where X' is intended to be equal to X, but it is not.
        \item[Availibility:] Whether or not we can guarantee success or failure (alternatively: every request returns a non-error response).
        \item[Partition tolerance]:  If a node goes down, will the system continue to operate?
    \end{description}

    The CAP theorem states that any big-data system can only acheive two out of tree letters (CA, CP or AP).
\end{definition}

\begin{proof}
Assume a system has two nodes: A and B. A and B cannot communicate.
We write data X to node A and B. Then, we write $X'$ to node A. Following that we want to read $X$ from node B.
If node A and B do not talk together, we will not achieve consistency (node B doesn't know X is updated in A).
This also means we struggle with availibility: $X'$ has not been written to both nodes.
If we do let node A and B talk together, then node B depends on A, so we are not parition tolerant.
Hence, achieving all three letters is not possible in this case.
\end{proof}

Q:\@ how can a join be affected by whether or not we sort the output?
\newline A:\@ dwa.

Q:\@ discuss joins in relation with parallelism?
\newline A:\@ ah. 

Q:\@ describe diffs in traditional DBMS and data streams.
\newline A:\@ ah. 

Q:\@ Main principles GFS
\newline A:\@ ah. 

Q:\@ explain use of histograms for optimization
\newline A:\@ ah.

Q:\@ explain use of histograms for optimization
\newline A:\@ ah.

Q:\@ explain use of histograms for optimization
\newline A:\@ ah.


Q:\@ Should file partitions always have the same size?
\newline A:\@ ah.

Q:\@ Discuss parallelism in light of normal queries to a DBMS
\newline A:\@ ah.

Q:\@ Discuss parallelism in light of normal queries to a DBMS
\newline A:\@ ah.

Q:\@ Explain a sliding window in streaming systems
\newline A:\@ ah.

\subsection{HDFS}
A file-storage system. The idea is to provide high availibility and 
