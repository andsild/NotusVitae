\section{Big Data}
\begin{definition}[CAP theorem]\label{def:captheorem}
    Working with Big Data involves dealing with inputs so large that conventional methods do not work.
    This also involves that we have to compromise, not opting for ``optimal'' solutions like RDMS.\
    We have 
    \begin{description}
        \item[Consistency:] Whether or not copies of data are the same across all nodes.
        For example, having a server in London with data X and another server in USA with data X', where X' is intended to be equal to X, but it is not.
        \item[Availibility:] Whether or not we can guarantee success or failure (alternatively: every request returns a non-error response).
        \item[Partition tolerance]:  If a node goes down, will the system continue to operate?
    \end{description}

    The CAP theorem states that any big-data system can only acheive two out of tree letters (CA, CP or AP).
\end{definition}

\begin{proof}
Assume a system has two nodes: A and B.  To achieve 
\end{proof}
