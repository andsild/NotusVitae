\section{Programming Security}
It's considered difficult to protect against social engineering.
Some of the problem is that you questioning everyone takes time,
training everyone takes time, and that since being sceptical takes more
time for employees, people arent' going to do it. The best way
to protect against it is to never trust anyone, and build autonomous systems.

The problem about centralizing data is that once a breach is made, the effects
can be more severe. If a small datasource is compromised occasionally, the
negative impacts might still be small enough to be handled. However,
it is easier making centralized systems safe against small crimes.
\newline

Passwords are good to authenticate, but they're easy to guess and 
people are not good at ensuring policies for passwords. When you ask for
secure passwords, people tend to store them in an unsafe fashion such that
the \textbf{security at the expense of usability is usability at the expense
of security}.

In general, protocols may be defeated byy changing the environment 
that they operate in. Most protocols make some assumptions about how
things work, as soon as this is modified, the protocol might break.

A cool form of attack: if a privileged user has ``.'' in his path, and you
could put in a piece of software there, like ``ls'', you could trick the admin
into executing it. Since the ``ls'' in ``.'' might be preferred, he will then launch
a program with elevated privileges, that you might have put there.
